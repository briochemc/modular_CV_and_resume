%-------------------------------------
\section{Research Interests}
%-------------------------------------

After spending the last few years investigating the ocean and its response to global warming, I am currently seeking new opportunities and looking for my next project.



My work sits at the forefront of ocean sciences and uses cutting-edge mathematical and numerical tools.

I spend most of my time thinking about the fascinating mechanisms that drive the cycles of carbon, nutrients, and oxygen in the global ocean.
To improve our understanding, I build numerical models of tracers in the ocean.
This requires engaging with a diverse range of research fields including biology, geology, chemistry, and physics, and deep knowledge of advanced mathematical and computational tools rooted in linear algebra, differential equations, Green functions, nonlinear phenomena, statistics, and optimisation, to mention a few.
My education as a mathematician and engineer helps me to develop new ideas and methods to tackle challenging questions in ocean sciences.


My PhD at UNSW was spent studying the global marine cycles of nutrients and iron, which control the ocean's fertility and the ``biological pump''.
This is a critically important area of research as these nutrients sustain all life in the ocean and their cycles are predicted to respond dramatically to climate change.
My first postdoc at UCI expanded on this research and was dedicated to biogeochemistry modelling and optimisation.
During that time I also developed several open-source packages that provide researchers and students with effective tools for investigating global marine biogeochemical cycles.
For my second postdoc, I focused on trace elements and their isotopes, such as nickel, neodymium, cadmium, and iron, which provide complementary constraints and shed light on unresolved questions about the past, current, and future of the oceans.
As a Research Associate at UNSW, I worked on the response of the biological pump and the response of the oxygen cycle to climate change.
The future of the oxygen cycle is yet another critical issue because our warming climate is driving the ocean to lose its oxygen, which is essential for marine life, with direct impacts for global food security.
Lately, I have started building ocean transport matrices for investigating marine Carbon Dioxide Removal (mCDR) for CSIRO's CarbonLock Future Science Platform.


I firmly believe that scientists have a duty to make sure we understand our environment well enough to prepare for its abrupt change and prevent the worst outcomes.
I commend my fellow sea-going oceanographers, who play a crucial role in this pursuit by providing us with an ever-growing set of observational data.
As a mathematically inclined oceanographer, I am committed to contribute to that endeavour by putting all my energy towards answering the critical science questions posed by our changing environment and towards educating the next generation of scientists that will eventually take over.
Despite my expertise, which continuously exposes me to the grim outlook of climate change, I am regularly amazed by the scientific discoveries that we make and the positive outcomes that we can achieve.
I remain scientifically fascinated by the complex interplay between the ocean, biology, and climate, and I look forward to making a difference by working alongside wonderful collaborators.

Such progress depends on the development of new, open-source, and user-friendly scientific software that is scalable from the simplest 0-dimensional models to complex high-resolution simulations.
Key to the success of these tools is composability.
Combining classical simulations with state-of-the-art software for, e.g., data assimilation, parameter optimization, uncertainty analysis, Bayesian inference, and machine learning will bring about impactful breakthroughs.
I hope that I can contribute to such efforts in my future appointments.
